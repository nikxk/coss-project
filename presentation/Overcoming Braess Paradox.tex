\documentclass[aspectratio=169]{beamer}
\mode<presentation>

\usetheme{Goettingen}
\usecolortheme{orchid}
\useinnertheme{circles}
\usecolortheme{whale}
\usefonttheme[onlymath]{serif}

\newcommand{\nameA}{Nikhil Sethukumar}
\newcommand{\nameB}{Vishnu Varadan}
\newcommand{\emailA}{nsethukumar@ethz.ch}
\newcommand{\emailB}{vvaradan@ethz.ch}
\newcommand{\projtitle}{Overcoming Braess Paradox}


\title{\projtitle}
\author{Nikhil Sethukumar \and Vishnu Varadan }
\date{\today}
\subtitle{Complex Social Systems, Agent Based Modelling \\ Fall 2022}
\institute{ETH Z\"urich}

\begin{document}

\begin{frame}

    \titlepage

\end{frame}

\begin{frame}
    \frametitle{Outline}

    \tableofcontents

\end{frame}

\section{What is Braess Paradox?}

\subsection{Refresher on Game Theory}

\begin{frame}
    \frametitle{A refresher on Game Theory}

    % Include a rock paper scissors game matrix

    % mention agents, actions, rewards, stages

    % What is NE? social cost and price of anarchy

\end{frame}

\subsection{The congestion game}

\begin{frame}
    \frametitle{Crossing the Grand Canal in Venice}

    % introduce with an example, the congestion game - add alley/canal image from trip

\end{frame}

\begin{frame}
    \frametitle{Modelling it mathematically}

    % present game matrix / example numbers and NE solution with and without bridge

\end{frame}

\section{Overcoming the paradox}

\subsection{The centralized dictator}

\begin{frame}
    \frametitle{Having a centralized dictator}

    % Present absolute optima, argue lots of info exchange needed to achieve

\end{frame}

\subsection{Networking approach}

\begin{frame}
    \frametitle{Networking and benefits}

    

\end{frame}

\subsection{Intelligent agents}

\section{Conclusion}

\subsection{Have we solved it?}

\subsection{Applications}

% \section*{References}

% \begin{frame}{References}

%     \bibliographystyle{unsrt}
%     \bibliography{References}
    
% \end{frame}

\end{document}