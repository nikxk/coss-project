\documentclass{beamer}
\usepackage[utf8]{inputenc}
\usepackage{amsmath}
\usetheme{Madrid}
\usecolortheme{seahorse}
\usefonttheme[onlymath]{serif}
\graphicspath{ {images/} }
\setbeamertemplate{enumerate items}[square]

\newcommand{\p}[2][1]{\begin{center}\includegraphics[width=#1\linewidth]{#2}\end{center}}

\title[Chaos and Atmospheric predictability]{Chaos theory and Atmospheric predictability}
\subtitle{ME5127 Term paper}
\author[Nikhil S]{Nikhil S \textcolor{darkgray}{(ME17B077)}}
\institute{IITM}
\date{\today}

\begin{document}

\frame{\titlepage}

\begin{frame}
{\textbf{A}periodic Map}
\only<1>{Let us consider a simple function:}
\[\mathrm{f}(x)=4x(1-x) ,\qquad \mathrm{f}: [0,1]\to[0,1]\]
\only<2-3,5>{
\begin{itemize}
\item<2-> $x=0.4$
\item<2-> $\mathrm{f}(x)=0.96$, $\mathrm{f}(\mathrm{f}(x))=0.1536$
\item<3-> $\{0.4,0.96,0.1536,0.520028,0.998395,0.00640774\}$
\item<5-> Now we also take \textcolor{yellow}{$x_0=0.40001$} alongside \textcolor{teal}{$x_0=0.4$}
\end{itemize}}
\only<4>{\p[0.85]{chinmath1}}
\only<6>{\p[0.77]{chinmath2}}
\end{frame}

\begin{frame}
{Chaos in a Double pendulum}
\only<1>{We considered a discretised system before, now we shall look at a continuous (\emph{in time}) system.}

\only<2->{
\begin{columns}
\column{0.3\linewidth}
\p{DPWork}
\column{0.7\linewidth}
\only<2>{
We use Lagrangian mechanics for the formulation:
\small
\begin{align*}
\mathcal{T} &= \frac{1}{2} l^2 m \left(2 \left(\theta _1'\right){}^2+\left(\theta _2'\right){}^2+2 \theta _2' \theta _1' \cos \left(\theta _1-\theta _2\right)\right)\\
\mathcal{V} &= -g l m \left(2 \cos \left(\theta _1\right)+\cos \left(\theta _2\right)\right)\\
\mathcal{L}&=\mathcal{T}-\mathcal{V}\\
\end{align*}
\normalsize
Now the Euler-Lagrange equations:
\[\frac{\mathrm{d}}{\mathrm{d}t}\left(\frac{\partial \mathcal{L}}{\partial q'}\right)-\frac{\partial \mathcal{L}}{\partial q}=0\]
with $q=\theta_1,\,\theta_2$.

And time is also rescaled as $t\to \sqrt{\frac{l}{g}}\tau$ in the equations of motion}

\only<3-4>{
The equations of motion are:
\tiny
\begin{align*}
\left(\theta _2'\right){}^2 \sin \left(\theta _1-\theta _2\right)+2 \theta _1''+\theta _2'' \cos \left(\theta _1-\theta _2\right)+2 \sin \left(\theta _1\right)&=0 \\
\left(\theta _1'\right){}^2 \left(-\sin \left(\theta _1-\theta _2\right)\right)+\theta _2''+\theta _1'' \cos \left(\theta _1-\theta _2\right)+\sin \left(\theta _2\right)&=0 
\end{align*}
\normalsize
\only<4>{
\vfill
We consider the following initial conditions:
\begin{table}
\centering
\begin{tabular}{|c |c c c c|}
\hline
\textsf{Initial conditions}& $\theta_1(0)$ & $\theta_2(0)$& $\theta_1'(0)$ & $\theta_2'(0)$\\\hline
System 1&$120^\circ$&$80^\circ$&0&0\\
System 2&$120.1^\circ$&$79.9^\circ$&0&0\\\hline
\end{tabular}
\end{table}
}
}
\only<5>
{\p{DPq1}
\p{DPq2}
}
\end{columns}}

\end{frame}

\begin{frame}
{Features of Chaos}
\begin{definition}
\textbf{Chaos} is aperiodic long-term behaviour in a deterministic system that exhibits sensitive dependence on initial conditions.
\end{definition}

\setbeamercovered{transparent}
\begin{enumerate}
\item<2-> Aperiodic long-term behaviour

\only<2>{\textcolor{darkgray}{There are trajectories
which do not settle down to fixed points, periodic orbits, or quasiperiodic orbits as $t\to\infty$}}
\item<3-> ``Deterministic"

\only<3>{\textcolor{darkgray}{The system has no random or noisy
inputs or parameters. The irregular behavior arises from the system’s
nonlinearity, rather than from noisy driving forces}}
\item<4-> Sensitive dependence on initial conditions

\only<4>{\textcolor{darkgray}{Nearby trajectories separate exponentially fast}}
\end{enumerate}


\setbeamercovered{invisible}
\end{frame}

\begin{frame}
{Lorenz's work}
\only<-4>{
\begin{itemize}
\item<1-> Edward Lorenz was a meteorologist at MIT. He noticed an anomaly in his work in 1961
\item<2-> He pursued the reason behind it, publishing a paper in 1962, \textbf{`Deterministic non-periodic flow'}
\item<3-> He studied a skeletal form of Saltzman's convection equations:
\begin{align*}
\frac{\partial}{\partial t} \nabla^{2} \psi &=-\frac{\partial\left(\psi, \nabla^{2} \psi\right)}{\partial(x, z)}+\nu \nabla^{4} \psi+g \alpha \frac{\partial \theta}{\partial x} \\
\frac{\partial}{\partial t}&=-\frac{\partial(\psi, \theta)}{\partial(x, z)}+\frac{\Delta T}{H} \frac{\partial \psi}{\partial x}+\kappa \nabla^{2} \theta
\end{align*}
\item<4-> Which, after much simplification, transformed to 
\begin{align*}
\dot{x}&=\sigma(y-x)\\
\dot{y}&=r x-y-x z\\
\dot{z}&=x y-b z
\end{align*}
with $\sigma,r,b>0$, all of which are parameters.
\end{itemize}
}

\only<5->
{
\begin{figure}
\centering
\includegraphics[width=0.45\linewidth]{LAttract1}
\includegraphics[width=0.45\linewidth]{LAttract2}

\caption{Lorenz attractors for $\sigma=10, b=8/3,r=28$ and initial conditions \textcolor{teal}{(0,1,0)} and \textcolor{yellow}{(1,0,0)}}
\label{LAttract}
\end{figure}
}
\end{frame}

\begin{frame}
{Atmospheric predictability}
\only<1>{
\begin{figure}
\includegraphics[width=0.7\textwidth]{Athirappally}
\caption{Athirapally Falls on 7-3-2021, \textit{Photo: Nikhil S}}
\end{figure}}

\only<2->
{
\only<2-3>{
\begin{itemize}
\item Even if we know the positions of all particles in the river, we can only know them to a finite accuracy
\item As time goes on, the errors multiply and increase exponentially
\end{itemize}}

\begin{block}<3->
{Lorenz concludes:}
When our results concerning the instability of non-periodic flow are applied to the atmosphere, which is
ostensibly nonperiodic, they indicate that prediction of
the 
\textit{sufficiently distant future is impossible by any
method, unless the present conditions are known exactly}. In view of the inevitable inaccuracy and incompleteness of weather observations, \textbf{precise very-long-range forecasting would seem to be non-existent}.
\end{block}
}
\only<4>{
\vfill 

The European Centre’s assessments suggested that the world saved billions of dollars each year from predictions that were statistically better than nothing. But beyond \underline{two or three days} the world’s best forecasts were speculative.\hfill \ldots [James Gleick, 1987]
}
\end{frame}

\begin{frame}
\vfill
\LARGE
\begin{center}
\includegraphics[width=0.4\textwidth]{End}
\\[1cm]
\includegraphics[width=0.27\textwidth]{Thanks}
\\[1cm]
\includegraphics[width=0.3\textwidth]{Name}
\end{center} 
\end{frame}

\end{document}
